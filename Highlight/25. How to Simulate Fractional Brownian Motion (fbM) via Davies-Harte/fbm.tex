\documentclass[11pt,a4paper]{article}
\usepackage[T1]{fontenc}
\usepackage[utf8]{inputenc}
\usepackage{lmodern}
\usepackage{amsmath,amssymb,amsfonts,mathtools,bm}
\usepackage[a4paper,margin=2.2cm]{geometry}
\usepackage{hyperref}
\hypersetup{colorlinks=true,linkcolor=black,urlcolor=blue,citecolor=black}
\numberwithin{equation}{section}
\allowdisplaybreaks

% ---------- Macros ----------
\newcommand{\E}{\mathbb{E}}
\newcommand{\R}{\mathbb{R}}
\newcommand{\N}{\mathcal{N}}
\newcommand{\Var}{\mathrm{Var}}
\newcommand{\Cov}{\mathrm{Cov}}
\newcommand{\1}{\mathbf{1}}

\begin{document}
\section*{Simulation du mouvement brownien fractionnaire — Méthode de Davies–Harte}

\textbf{I. Discrétisation du fBm}

On considère un Brownien fractionnaire sur la période $T$, avec un pas $\Delta t = \frac{T}{N}$.
On définit :
\[
t_k = k \, \Delta t, \quad k = 0, \dots, N
\]
et
\[
B_H(t_k) = \sum_{j=0}^k \xi_j,
\]
où $\xi_j$ est un bruit gaussien corrélé (fractional Gaussian noise).

\vspace{1em}

\textbf{II. Covariance du bruit fractionnaire}

On note :
\[
\gamma(k) = \Cov(X_t, X_{t+k}) = \E[X_t X_{t+k}] 
= \tfrac{1}{2}\left(|k+1|^{2H} + |k-1|^{2H} - 2|k|^{2H}\right)
\]

On construit le vecteur :
\[
c = [\gamma(0), \gamma(1), \dots, \gamma(N), \gamma(N-1), \dots, \gamma(1)].
\]

\vspace{1em}

\textbf{III. Diagonalisation par FFT}

On calcule :
\[
\Lambda = \mathrm{FFT}(c)
\]
et on pose la matrice de Fourier :
\[
F_{jk} = e^{-2i\pi jk / M}, \quad M = 2N
\]

Alors :
\[
C = F^{-1} \, \mathrm{diag}(\Lambda) \, F
\]

On génère le vecteur aléatoire :
\[
Z = (Z_0, Z_1, \dots, Z_{2N-1}) \in \C^{2N}, \quad Z_k \sim \N(0,1) + i \N(0,1)
\]
avec la symétrie $Z_{2N-k} = \overline{Z_k}$ pour assurer la réalité du signal.

\vspace{1em}

\textbf{IV. Génération du bruit fractionnaire}

On définit :
\[
Y = F^{-1} \left( \sqrt{\Lambda} \odot Z \right),
\]
puis on garde la partie réelle :
\[
(X_0, X_1, \dots, X_{N-1}) = \Re(Y_0, Y_1, \dots, Y_{N-1})
\]
qui correspond au \emph{fractional Gaussian noise}.

\vspace{1em}

\textbf{V. Reconstruction du fBm}

On cumule :
\[
B_H(t_k) = \sum_{j=0}^{k-1} X_j,
\]
avec $B_H(0)=0$.  
La variance est correctement ajustée en multipliant par $(\Delta t)^H$ :
\[
B_H(t_k) = (\Delta t)^H \sum_{j=0}^{k-1} X_j.
\]

\vspace{1em}

\textbf{Résumé}
\begin{itemize}
    \item On construit la covariance théorique $\gamma(k)$ du bruit fractionnaire.
    \item On applique une FFT sur le vecteur $c$ étendu.
    \item On génère un vecteur gaussien complexe symétrique.
    \item On multiplie par $\sqrt{\Lambda}$ et on applique l’IFFT.
    \item On récupère la partie réelle pour obtenir le bruit.
    \item On cumule pour obtenir le mouvement brownien fractionnaire $B_H(t)$.
\end{itemize}

\end{document}
